\documentclass{article}
\PassOptionsToPackage{numbers,compress}{natbib}
\usepackage[final]{template22}
% Common packages
\usepackage[utf8]{inputenc} % allow utf-8 input
\usepackage[T1]{fontenc}    % use 8-bit T1 fonts
\usepackage{microtype}
\usepackage{times}
\usepackage{graphicx}
\usepackage{amsmath,amssymb,mathbbol}
% \usepackage{algorithmic}
% \usepackage[linesnumbered,ruled,vlined]{algorithm2e}
\usepackage{acronym}
\usepackage{enumitem}
\usepackage[pagebackref=true,breaklinks=true,colorlinks]{hyperref}
\usepackage{balance}
\usepackage{xspace}
\usepackage{setspace}
\usepackage[skip=3pt,font=small]{subcaption}
\usepackage[skip=3pt,font=small]{caption}
\usepackage[dvipsnames]{xcolor}
\usepackage[capitalise]{cleveref}
\usepackage{booktabs,tabularx,colortbl,multirow,array,makecell}
% \usepackage{overpic,wrapfig}

\usepackage{fancyhdr}
\hypersetup{pdfencoding=auto,colorlinks=true,allcolors=black}
\renewcommand{\headrulewidth}{0.5pt}
\renewcommand{\footrulewidth}{0pt}
\fancyhf{}
\fancyhead[C]{}
\fancyhead[C]{}
\fancyfoot[C]{\thepage}

% Handy shorthand
\makeatletter
\DeclareRobustCommand\onedot{\futurelet\@let@token\@onedot}
\def\@onedot{\ifx\@let@token.\else.\null\fi\xspace}
\def\eg{\emph{e.g}\onedot} 
\def\Eg{\emph{E.g}\onedot}
\def\ie{\emph{i.e}\onedot} 
\def\Ie{\emph{I.e}\onedot}
\def\cf{\emph{c.f}\onedot} 
\def\Cf{\emph{C.f}\onedot}
\def\etc{\emph{etc}\onedot} 
\def\vs{\emph{vs}\onedot}
\def\wrt{w.r.t\onedot} 
\def\dof{d.o.f\onedot}
\def\etal{\emph{et al}\onedot}
\makeatother

\definecolor{gray}{gray}{0.9}

% Handy math ops
\DeclareMathOperator*{\argmax}{arg\,max}
\DeclareMathOperator*{\argmin}{arg\,min}
\newcommand{\norm}[1]{\left\Vert #1 \right\Vert}

% % Spacing
\frenchspacing
% \medmuskip=2mu   % reduce spacing around binary operators
% \thickmuskip=3mu % reduce spacing around relational operators
% \setlength{\abovedisplayskip}{3pt}
% \setlength{\belowdisplayskip}{3pt}
% \setlength{\abovecaptionskip}{3pt}
% \setlength{\belowcaptionskip}{3pt}
\setlength\floatsep{0.5\baselineskip plus 3pt minus 2pt}
\setlength\textfloatsep{0.5\baselineskip plus 3pt minus 2pt}
\setlength\dbltextfloatsep{0.5\baselineskip plus 3pt minus 2pt}
\setlength\intextsep{0.5\baselineskip plus 3pt minus 2pt}

\makeatletter
\renewcommand{\paragraph}{%
  \@startsection{paragraph}{4}%
  {\z@}{0ex \@plus 0ex \@minus 0ex}{-1em}%
  {\hskip\parindent\normalfont\normalsize\bfseries}%
}
\makeatother

% Graphics path
\graphicspath{{figures/}}

% Clever references
\crefname{algorithm}{Alg.}{Algs.}
\Crefname{algorithm}{Algorithm}{Algorithms}
\crefname{section}{Sec.}{Secs.}
\Crefname{section}{Section}{Sections}
\crefname{table}{Tab.}{Tabs.}
\Crefname{table}{Table}{Tables}
\crefname{figure}{Fig.}{Fig.}
\Crefname{figure}{Figure}{Figure}

% Acronym
\acrodef{pku}[PKU]{Peking University}

\title{Translate books for the visually impaired}

\author{%
  Xiang Wang\\
  Department of Artificial intelligence\\
  Peking University\\
  \texttt{2100013146@stu.pku.edu.cn} \\
  \And
  Xin Hao \\
  Department of Artificial intelligence\\
  Peking University\\
  \texttt{email} \\
}

\begin{document}
\maketitle

\begin{abstract}
The abstract must be limited to one paragraph; see an annotated \href{https://cbs.umn.edu/sites/cbs.umn.edu/files/public/downloads/Annotated_Nature_abstract.pdf}{Nature abstract} as the guideline. In particular, you will need one or two sentences providing a basic introduction to the field, comprehensible to a scientist in any discipline. Next, provide two to three sentences of more detailed background, comprehensible to scientists in related disciplines. The next two sentences are probably the most essential one in the abstract: One sentence clearly stating the general problem being addressed by this particular study, and one sentence summarising the main result (with the words ``here we show'' or their equivalent). The rest are detailed analysis of results: (i) Two or three sentences explaining what the main result reveals in direct comparison to what was thought to be the case previously, or how the main result adds to previous knowledge. (ii) One or two sentences to put the results into a more general context. (iii) Two or three sentences to provide a broader perspective, readily comprehensible to a scientist in any discipline, may be included in the first paragraph if the editor considers that the accessibility of the paper is significantly enhanced by their inclusion. Under these circumstances, the length of the paragraph can be up to 300 words.
\end{abstract}



\bibliographystyle{plainnat}
\bibliography{reference}

\appendix

\section{Appendix}

Optionally include extra information (complete proofs, additional experiments and plots) in the appendix. This section will often be part of the supplemental material.

\end{document}